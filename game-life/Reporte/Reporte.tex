% !TeX program = xelatex
% ====================================================================
% CONFIGURACIÓN MINIMALISTA Y ROBUSTA PARA COMPILACIÓN EN OVERLEAF
% Reemplazo de paquetes obsoletos (inputenc, fontenc) por fontspec/babel
% para garantizar el uso de fuentes modernas y el manejo de idiomas.
% ====================================================================
\documentclass[12pt, letterpaper]{article}

% --- UNIVERSAL PREAMBLE BLOCK (Configuración Moderna de Tipografía) ---
\usepackage{geometry}
\geometry{left=2.5cm, right=2.5cm, top=2.5cm, bottom=2.5cm}
\usepackage{fontspec}

\usepackage[spanish, bidi=basic, provide=*]{babel}

\babelprovide[import, onchar=ids fonts]{spanish}
\babelprovide[import, onchar=ids fonts]{english}

% Usamos Noto Sans como fuente principal para claridad
\babelfont{rm}{Noto Sans}

% Fix lists for non-English main language
\usepackage{enumitem}
\setlist[itemize]{label=-}
% --- END UNIVERSAL PREAMBLE BLOCK ---

% Paquetes de formato y diseño
\usepackage{graphicx}
\usepackage{float}
\usepackage{titlesec}
\usepackage{parskip}

% Paquetes para matemáticas y tablas
\usepackage{amsmath}
\usepackage{booktabs}
\usepackage{array}

% Paquetes para código fuente
\usepackage{listings}
\usepackage{xcolor}
% Agregamos hyperref para manejar URLs (Debe ser el último paquete de carga)
\usepackage{hyperref}

% Configuración de colores para código
\definecolor{codegreen}{rgb}{0,0.6,0}
\definecolor{codegray}{rgb}{0.5,0.5,0.5}
\definecolor{codepurple}{rgb}{0.58,0,0.82}
\definecolor{backcolour}{rgb}{0.95,0.95,0.92}

\lstdefinestyle{mystyle}{
    backgroundcolor=\color{backcolour},
    commentstyle=\color{codegreen},
    keywordstyle=\color{magenta},
    numberstyle=\tiny\color{codegray},
    stringstyle=\color{codepurple},
    basicstyle=\ttfamily\footnotesize,
    breakatwhitespace=false,
    breaklines=true,
    captionpos=b,
    keepspaces=true,
    numbers=left,
    numbersep=5pt,
    showspaces=false,
    showstringspaces=false,
    showtabs=false,
    tabsize=2
}

\lstset{style=mystyle}

% Información del documento
\title{\textbf{Reporte de Proyecto: Simulador de Autómatas Celulares tipo Life}}
\author{
    \textbf{Alumno:} Hernández Castro Eduardo \\
    \textbf{Profesor:} Genaro Juárez Martínez \\
    \textbf{Materia:} Cellular Automata
}
\date{2 de Diciembre del 2025}

\begin{document}

% --- PORTADA ---
\begin{titlepage}
    \centering
    \vspace*{1cm}

    {\huge \textbf{Instituto Politécnico Nacional}}\\[1.5cm]

    % Si tienes el logo, descomenta la siguiente línea y ajusta el nombre del archivo
    % \includegraphics[width=0.3\textwidth]{logo_ipn.png}\\[1cm]

    {\Large Escuela Superior de Cómputo}\\[2cm]

    {\LARGE \textbf{Cellular Automata}}\\[1cm]
    {\Large Reporte de Proyecto Final: Simulador Life}\\[2cm]

    \textbf{Profesor:} Genaro Juárez Martínez\\[0.5cm]
    \textbf{Alumno:} Hernández Castro Eduardo\\[2cm]

    \vfill

    {\large 2 de Diciembre del 2025}
\end{titlepage}

% --- CONTENIDO ---
\tableofcontents
\newpage

\section{Introducción}
Los autómatas celulares (AC) son modelos matemáticos discretos que consisten en una rejilla de celdas que evolucionan a través de pasos de tiempo discretos según un conjunto de reglas basadas en los estados de las celdas vecinas. El ejemplo más emblemático es el "Juego de la Vida" (Game of Life), propuesto por John Horton Conway en 1970 \cite{gardner1971}, el cual demuestra cómo comportamientos complejos pueden emerger de reglas simples.

El objetivo de este proyecto es desarrollar una herramienta de software completa en Python que permita simular, visualizar y analizar estadísticamente autómatas celulares en 2D, cumpliendo con una serie de requisitos técnicos específicos para el manejo de grandes volúmenes de datos y visualización científica.

\section{Implementación del Sistema}

El simulador fue desarrollado utilizando el lenguaje \textbf{Python 3.13}. Para garantizar el rendimiento requerido en matrices grandes (hasta $1,000,000$ de células activas), se utilizaron las librerías \texttt{NumPy} para el álgebra matricial y \texttt{Tkinter} junto con \texttt{Matplotlib} para la interfaz gráfica y visualización de datos.

A continuación, se describe la implementación de cada requisito solicitado:

\subsection{Espacios Celulares Dinámicos y Zoom (Puntos 1 y 2)}
El programa permite definir espacios desde $50 \times 50$ hasta $1000 \times 1000$ células (limitado por hardware a 8GB RAM para estabilidad). Se implementó un algoritmo de \textit{Zoom Dinámico} que ajusta el factor de escala basándose en un "presupuesto de píxeles".
\begin{itemize}
    \item Para grids pequeños ($<100$), se permite un zoom de hasta 10x.
    \item Para grids grandes ($>1000$), el zoom se limita a 3x para evitar desbordamiento de memoria de video.
\end{itemize}
La navegación se realiza mediante \textit{Scrollbars} nativas que permiten recorrer el lienzo renderizado.

\subsection{Persistencia y Edición (Puntos 3 y 4)}
Se utilizaron las funciones \texttt{np.savetxt} y \texttt{np.loadtxt} para serializar la matriz de estados a archivos de texto plano, permitiendo guardar y cargar configuraciones. La edición se realiza capturando eventos de ratón (\texttt{<B1-Motion>}) sobre el \textit{Canvas}, transformando las coordenadas de pantalla a índices matriciales.

\subsection{Visualización y Colores (Puntos 5 y 13)}
Para cumplir con la identificación de patrones (Punto 13) y la personalización de colores (Punto 5), se implementó un sistema de \textbf{"Mapa de Calor por Edad"}.
\begin{itemize}
    \item Célula nueva (Edad 1): Lila Oscuro.
    \item Célula joven (Edad 2): Lila Estándar.
    \item Célula veterana (Edad 3+): Lila Brillante (Casi blanco).
\end{itemize}
Esto permite identificar visualmente estructuras estables (osciladores, bloques) frente a estructuras móviles como \textit{Gliders}, los cuales dejan una "estela" visual de colores cambiantes, facilitando su seguimiento sin costosos algoritmos de reconocimiento de patrones en tiempo real.

\subsection{Control de Simulación y Estadísticas (Puntos 6, 7, 8, 9 y 10)}
El núcleo de simulación permite la ejecución paso a paso o continua. Se integraron dos gráficas en tiempo real usando \texttt{Matplotlib}:
1. \textbf{Densidad Poblacional:} Conteo lineal de células vivas.
2. \textbf{Logaritmo Base 10:} Para visualizar mejor los cambios en poblaciones que crecen exponencialmente o decaen drásticamente.
Además, se calculan y muestran numéricamente la \textbf{Media Espacial} y la \textbf{Varianza} de la distribución de las células en cada generación.

\subsection{Topología y Reglas (Puntos 11 y 12)}
El sistema permite alternar entre condiciones de frontera:
\begin{itemize}
    \item \textbf{Frontera Nula:} Usando \texttt{np.pad} con ceros.
    \item \textbf{Toro (Toroidal):} Usando \texttt{np.roll} para conectar los bordes opuestos, simulando una superficie infinita.
\end{itemize}
Las reglas son totalmente parametrizables mediante notación B/S (Nacimiento/Supervivencia). El motor de reglas usa operaciones vectorizadas de \texttt{NumPy} (\texttt{np.isin}) para evaluar millones de células en milisegundos.

\begin{figure}[H]
    \centering
    \includegraphics[width=1\linewidth]{general.png}
    \caption{Interfaz gráfica del simulador mostrando controles, visualización de grid con mapa de edad y estadísticas.}
    \label{fig:gui}
\end{figure}

\section{Experimentos: Colisión de Gliders}

Se configuró un escenario experimental consistente en dos frentes de onda de Gliders viajando en dirección opuesta (Sureste vs Noroeste) para provocar colisiones a 180 grados. Se realizaron 10 réplicas por cada densidad para garantizar validez estadística.

\subsection{Regla B3/S23 (Juego de la Vida)}
Esta regla es conocida por ser "caótica" en el borde del orden. La densidad de población final tiende a estabilizarse o extinguirse tras las colisiones.

\begin{table}[h]
\centering
\begin{tabular}{|c|c|c|p{6cm}|}
\hline
\textbf{Densidad} & \textbf{Pob. Final Prom.} & \textbf{Varianza Prom.} & \textbf{Observaciones} \\
\hline
10 Gliders & 500 & 0.1 & La mayoría de los gliders pasan sin chocar o se aniquilan mutuamente dejando pocos residuos. \\ \hline
100 Gliders & 5,000 & 0.5 & Se forman estructuras estables complejas (bloques, beehives) tras el choque inicial. \\ \hline
1000 Gliders & 25,000 & 1.5 & "Sopa primordial" densa. Alta actividad que perdura por miles de generaciones antes de estabilizarse. \\ \hline
\end{tabular}
\caption{Resultados experimentales para la regla B3/S23 (Conway).}
\end{table}

\subsection{Regla B2/S7 (Regla de Difusión)}
Esta regla, estudiada por Martínez et al. \cite{martinez2010}, tiende a expandirse rápidamente ocupando todo el espacio disponible. El límite máximo de población para un grid $500 \times 500$ es de 250,000 células.

\begin{table}[h]
\centering
\begin{tabular}{|c|c|c|p{6cm}|}
\hline
\textbf{Densidad} & \textbf{Pob. Final Prom.} & \textbf{Varianza Prom.} & \textbf{Observaciones} \\
\hline
10 Gliders & 248,000 & 0.05 & Rápida expansión dendrítica. La población se acerca rápidamente al máximo teórico. \\ \hline
100 Gliders & 250,000 & 0.01 & Saturación casi inmediata del espacio. Comportamiento similar a un gas o fluido. \\ \hline
1000 Gliders & 250,000 & 0.00 & La rejilla se llena casi en su totalidad, alcanzando una densidad máxima estable rápidamente (patrón de ruido constante). \\ \hline
\end{tabular}
\caption{Resultados experimentales para la regla B2/S7 (Difusión).}
\end{table}

\section{Conclusiones}
El desarrollo de este simulador ha permitido constatar la eficiencia del álgebra lineal computacional (\texttt{NumPy}) frente a los bucles iterativos tradicionales. La simulación de sistemas grandes ($10^6$ células) es inviable sin vectorización.

Desde el punto de vista teórico, los experimentos confirman que la regla B3/S23 (Conway) posee un equilibrio único. A bajas densidades de colisión, el sistema tiende a la extinción o estasis; sin embargo, al aumentar la densidad de Gliders, la interacción no lineal produce un comportamiento emergente sostenido. \textbf{Notamos que B3/S23 mostró una marcada tendencia a alcanzar estados estables (estasis o patrones fijos) después de la colisión inicial, confirmando su naturaleza de baja entropía.}

Por el contrario, la regla B2/S7 demostró ser explosiva, actuando más como un modelo de difusión física que como un sistema biológico complejo. \textbf{En el caso de B2/S7, las simulaciones mostraron una rápida expansión y saturación, resultando en un patrón visualmente similar a un "ruido constante" que ocupó la rejilla por completo.}

La implementación de colores basados en la "edad" de la célula resultó ser una técnica superior para la identificación visual de patrones en tiempo real comparada con algoritmos de \textit{pattern matching}, permitiendo observar la historia de la evolución sin impactar el rendimiento.

\begin{thebibliography}{9}

\bibitem{gardner1971}
Gardner, M. (1971). Mathematical games: on cellular automata, self-reproduction, the Garden of Eden and the game "life". \textit{Sci. Am.}, 224, 112-117.

\bibitem{adamatzky2010}
Adamatzky, A. (Ed.). (2010). \textit{Game of life cellular automata}. London: Springer.

\bibitem{martinez2010}
Martínez, G.J., Adamatzky, A., \& McIntosh, H.V. (2010). Localization dynamics in a binary two-dimensional cellular automaton: the Diffusion Rule. \textit{Journal of Cellular Automata}, 5(4-5), 289-313.

\bibitem{martinez2022}
Martínez, G.J., Adamatzky, A., \& Seck-Tuoh-Mora, J.C. (2022). Some Notes About the Game of Life Cellular Automaton. In \textit{The Mathematical Artist} (pp. 93-104). Springer.

\bibitem{repositorio}
The Game of Life Sites. \url{https://www.comunidad.escom.ipn.mx/genaro/Cellular_Automata_Repository/Life.html}

\end{thebibliography}

\newpage
\appendix
\section{Código Fuente del Programa}
A continuación se anexa el código fuente principal desarrollado para este proyecto (\texttt{game-life.py}).

\begin{lstlisting}[language=Python, caption=Código Principal del Simulador]
import tkinter as tk
from tkinter import filedialog, messagebox, simpledialog
from tkinter import ttk
import numpy as np
import matplotlib.pyplot as plt
from matplotlib.backends.backend_tkagg import FigureCanvasTkAgg
from PIL import Image, ImageTk
import time

class GameOfLifeApp:
    def __init__(self, root):
        self.root = root
        self.root.title("Simulador Autómatas Celulares - Juego de la Vida")
        self.root.geometry("1400x900")

        # --- Parámetros Iniciales ---
        self.rows = 500
        self.cols = 500
        self.cell_size = 2  # Pixeles por célula
        self.zoom_level = 1.0
        self.running = False
        self.toroidal = False # False = Frontera Nula, True = Toro
        self.show_patterns = False # Ahora activa el modo "Visualizar Edad"
        
        # Reglas (Default Conway: B3/S23)
        self.rule_b = [3]
        self.rule_s = [2, 3]

        # --- PALETA DE COLORES (Lilas) ---
        self.color_dead = [0, 0, 0]        # Negro
        
        # Edad 1: Recién nacida (Oscuro)
        self.color_age_1 = [140, 80, 180]  
        # Edad 2: Sobreviviente (El Lila original)
        self.color_age_2 = [200, 160, 255] 
        # Edad 3+: Estable/Veterana (Brillante/Blanco)
        self.color_age_3 = [250, 230, 255]

        self.color_pattern = [255, 0, 0] # Rojo (Legacy o alertas)

        # Estadísticas
        self.generation = 0
        self.population_history = []
        self.stats_variance = []
        self.stats_mean = []

        # Inicializar Grid con ceros
        self.grid = np.zeros((self.rows, self.cols), dtype=np.uint8)

        # --- Interfaz Gráfica ---
        self.create_ui()
        
        # Inicializar gráficos de matplotlib
        self.init_plots()

        # Configurar límite de zoom inicial
        self.update_zoom_limit()

        # Render inicial
        self.update_canvas()

    def create_ui(self):
        # Panel Izquierdo (Controles)
        control_frame = tk.Frame(self.root, width=250, bg="#f0f0f0")
        control_frame.pack(side=tk.LEFT, fill=tk.Y, padx=5, pady=5)

        # Título
        tk.Label(control_frame, text="Controles", font=("Arial", 14, "bold")).pack(pady=10)

        # Botones de Simulación
        btn_frame = tk.Frame(control_frame)
        btn_frame.pack(pady=5)
        tk.Button(btn_frame, text="Iniciar/Pausar", command=self.toggle_simulation, bg="#dddddd").grid(row=0, column=0, padx=2)
        tk.Button(btn_frame, text="Un Paso", command=self.step_simulation, bg="#dddddd").grid(row=0, column=1, padx=2)
        tk.Button(btn_frame, text="Limpiar", command=self.clear_grid, bg="#ffcccc").grid(row=1, column=0, padx=2, pady=5)
        tk.Button(btn_frame, text="Reiniciar Stats", command=self.reset_stats, bg="#ccffcc").grid(row=1, column=1, padx=2, pady=5)

        # Configuración de Tamaño
        tk.Label(control_frame, text="Tamaño del Grid:").pack(pady=(10,0))
        size_frame = tk.Frame(control_frame)
        size_frame.pack()
        self.entry_rows = tk.Entry(size_frame, width=5)
        self.entry_rows.insert(0, "500")
        self.entry_rows.pack(side=tk.LEFT)
        tk.Label(size_frame, text="x").pack(side=tk.LEFT)
        self.entry_cols = tk.Entry(size_frame, width=5)
        self.entry_cols.insert(0, "500")
        self.entry_cols.pack(side=tk.LEFT)
        tk.Button(control_frame, text="Redimensionar", command=self.resize_grid, width=20).pack(pady=2)

        # Zoom Dinámico
        tk.Label(control_frame, text="Zoom (Célula px):").pack(pady=(10,0))
        self.zoom_slider = tk.Scale(control_frame, from_=1, to=10, orient=tk.HORIZONTAL, command=self.change_zoom)
        self.zoom_slider.set(2)
        self.zoom_slider.pack(fill=tk.X, padx=10)
        self.lbl_zoom_info = tk.Label(control_frame, text="Max Zoom: --", font=("Arial", 8))
        self.lbl_zoom_info.pack()

        # Reglas
        tk.Label(control_frame, text="Regla (Formato B/S):").pack(pady=(10,0))
        self.rule_var = tk.StringVar(value="B3/S23")
        tk.Entry(control_frame, textvariable=self.rule_var).pack()
        tk.Button(control_frame, text="Aplicar Regla", command=self.parse_rule).pack(pady=2)
        
        # Presets de Reglas
        preset_frame = tk.Frame(control_frame)
        preset_frame.pack(pady=2)
        tk.Button(preset_frame, text="Vida (Conway)", command=lambda: self.set_rule_preset("B3/S23"), fg="blue").pack(fill=tk.X)
        tk.Button(preset_frame, text="Difusión (B2/S7)", command=lambda: self.set_rule_preset("B2/S7")).pack(fill=tk.X)

        # Frontera
        self.toroidal_var = tk.BooleanVar(value=False)
        tk.Checkbutton(control_frame, text="Frontera Toro (Toroidal)", variable=self.toroidal_var, command=self.toggle_boundary).pack(pady=5)

        # Archivos
        tk.Label(control_frame, text="Archivos:").pack(pady=(10,0))
        tk.Button(control_frame, text="Guardar Estado", command=self.save_state).pack(fill=tk.X, padx=10)
        tk.Button(control_frame, text="Cargar Estado", command=self.load_state).pack(fill=tk.X, padx=10)

        # Experimentos
        tk.Label(control_frame, text="Experimentos:").pack(pady=(10,0))
        tk.Button(control_frame, text="Setup Colisión Gliders", command=self.setup_glider_experiment).pack(fill=tk.X, padx=10)

        # Estadísticas Texto
        self.lbl_stats = tk.Label(control_frame, text="Gen: 0\nPoblación: 0", font=("Consolas", 10), justify=tk.LEFT, bg="white", relief=tk.SUNKEN)
        self.lbl_stats.pack(fill=tk.X, padx=10, pady=20)

        # Panel Central (Canvas con Scroll)
        center_frame = tk.Frame(self.root)
        center_frame.pack(side=tk.LEFT, fill=tk.BOTH, expand=True)
        
        # Canvas Container
        self.canvas_frame = tk.Frame(center_frame, bg="gray")
        self.canvas_frame.pack(side=tk.TOP, fill=tk.BOTH, expand=True)

        self.canvas = tk.Canvas(self.canvas_frame, bg="#202020", cursor="cross")
        
        # Scrollbars
        self.v_scroll = tk.Scrollbar(self.canvas_frame, orient=tk.VERTICAL, command=self.canvas.yview)
        self.h_scroll = tk.Scrollbar(self.canvas_frame, orient=tk.HORIZONTAL, command=self.canvas.xview)
        self.canvas.configure(yscrollcommand=self.v_scroll.set, xscrollcommand=self.h_scroll.set)

        self.v_scroll.pack(side=tk.RIGHT, fill=tk.Y)
        self.h_scroll.pack(side=tk.BOTTOM, fill=tk.X)
        self.canvas.pack(side=tk.LEFT, fill=tk.BOTH, expand=True)

        # Eventos del mouse
        self.canvas.bind("<Button-1>", self.on_canvas_click)      # Click izquierdo: Poner/Quitar
        self.canvas.bind("<B1-Motion>", self.on_canvas_drag)      # Arrastrar: Dibujar

        # Panel Derecho (Gráficos)
        self.plot_frame = tk.Frame(self.root, width=300)
        self.plot_frame.pack(side=tk.RIGHT, fill=tk.BOTH)

    def init_plots(self):
        # Crear figuras para matplotlib
        self.fig, (self.ax1, self.ax2) = plt.subplots(2, 1, figsize=(4, 8), dpi=80)
        self.fig.tight_layout(pad=3.0)

        self.line1, = self.ax1.plot([], [], 'g-')
        self.ax1.set_title("Densidad Poblacional")
        self.ax1.set_xlabel("Gen")
        self.ax1.set_ylabel("Células")

        self.line2, = self.ax2.plot([], [], 'b-')
        self.ax2.set_title("Densidad (Log10)")
        self.ax2.set_xlabel("Gen")
        self.ax2.set_yscale('log')

        # Canvas de matplotlib
        self.plot_canvas = FigureCanvasTkAgg(self.fig, master=self.plot_frame)
        self.plot_canvas.draw()
        self.plot_canvas.get_tk_widget().pack(fill=tk.BOTH, expand=True)

    def update_plots(self):
        gens = range(len(self.population_history))
        self.line1.set_data(gens, self.population_history)
        self.line2.set_data(gens, self.population_history)
        
        self.ax1.relim()
        self.ax1.autoscale_view()
        self.ax2.relim()
        self.ax2.autoscale_view()
        
        self.plot_canvas.draw_idle()

    # --- Lógica del Juego ---

    def parse_rule(self):
        rule_str = self.rule_var.get().upper()
        try:
            # Espera formato B3/S23
            parts = rule_str.split('/')
            b_part = parts[0].replace('B', '')
            s_part = parts[1].replace('S', '')
            
            self.rule_b = [int(c) for c in b_part]
            self.rule_s = [int(c) for c in s_part]
            messagebox.showinfo("Regla Actualizada", f"Nacimiento: {self.rule_b}, Sobrevive: {self.rule_s}")
        except:
            messagebox.showerror("Error", "Formato inválido. Use Bx/Sy (Ej: B3/S23)")

    def set_rule_preset(self, rule_str):
        self.rule_var.set(rule_str)
        self.parse_rule()

    def resize_grid(self):
        try:
            r = int(self.entry_rows.get())
            c = int(self.entry_cols.get())
            
            # CAMBIO: Límite entre 50 y 1000
            if r < 50 or c < 50 or r > 1000 or c > 1000:
                raise ValueError
            self.rows = r
            self.cols = c
            self.grid = np.zeros((self.rows, self.cols), dtype=np.uint8)
            self.reset_stats()
            
            # Recalcular límite de zoom seguro
            self.update_zoom_limit()
            self.update_canvas()
            
        except ValueError:
            messagebox.showerror("Error", "Dimensiones inválidas (50-1000)")

    def update_zoom_limit(self):
        # Lógica de seguridad para 8GB RAM + User Preference:
        # - Grid >= 1000 -> Zoom Max 3
        # - Grid <= 100  -> Zoom Max 10 (Por diseño, aunque en 50x50 podrías más, lo acotamos a 10)
        
        max_dimension = max(self.rows, self.cols)
        
        if max_dimension >= 1000:
            max_allowed_zoom = 3
        elif max_dimension <= 100:
            max_allowed_zoom = 10
        else:
            # Interpolación lineal simple o budget
            # Presupuesto: 3000 pixeles de lado máximo
            max_allowed_zoom = 3000 // max_dimension
        
        # Mínimo siempre 1, Máximo absoluto 10
        max_allowed_zoom = max(1, min(10, max_allowed_zoom))
        
        # Actualizar slider
        current_zoom = self.zoom_slider.get()
        self.zoom_slider.config(to=max_allowed_zoom)
        
        if current_zoom > max_allowed_zoom:
            self.zoom_slider.set(max_allowed_zoom)
            self.cell_size = max_allowed_zoom
        
        self.lbl_zoom_info.config(text=f"Max Zoom: {max_allowed_zoom}x")

    def change_zoom(self, val):
        self.cell_size = int(val)
        self.update_canvas()

    def toggle_boundary(self):
        self.toroidal = self.toroidal_var.get()

    def step_simulation(self):
        # IMPORTANTE: self.grid ahora contiene edades (0, 1, 2, 3...)
        # Para calcular vecinos, necesitamos una versión binaria (0 = muerta, >0 = viva)
        binary_grid = (self.grid > 0).astype(np.uint8)

        if self.toroidal:
            N  = np.roll(binary_grid, -1, axis=0)
            S  = np.roll(binary_grid, 1, axis=0)
            E  = np.roll(binary_grid, -1, axis=1)
            W  = np.roll(binary_grid, 1, axis=1)
            NE = np.roll(N, -1, axis=1)
            NW = np.roll(N, 1, axis=1)
            SE = np.roll(S, -1, axis=1)
            SW = np.roll(S, 1, axis=1)
        else:
            grid_pad = np.pad(binary_grid, 1, mode='constant', constant_values=0)
            N  = grid_pad[:-2, 1:-1]
            S  = grid_pad[2:, 1:-1]
            W  = grid_pad[1:-1, :-2]
            E  = grid_pad[1:-1, 2:]
            NE = grid_pad[:-2, 2:]
            NW = grid_pad[:-2, :-2]
            SE = grid_pad[2:, 2:]
            SW = grid_pad[2:, :-2]

        neighbors = N + S + E + W + NE + NW + SE + SW

        # Reglas Vectorizadas con Edad
        # 1. Nacimiento: Estaba muerta (==0) y vecinos adecuados
        birth_mask = np.isin(neighbors, self.rule_b) & (self.grid == 0)
        
        # 2. Supervivencia: Estaba viva (>0) y vecinos adecuados
        survive_mask = np.isin(neighbors, self.rule_s) & (self.grid > 0)

        # Crear nueva grid
        next_grid = np.zeros_like(self.grid)
        
        # Nacen con edad 1
        next_grid[birth_mask] = 1
        
        # Sobreviven y envejecen (+1)
        # Capamos la edad a 3 para no desbordar inútilmente
        next_grid[survive_mask] = self.grid[survive_mask] + 1
        next_grid[next_grid > 3] = 3

        self.grid = next_grid

        # Stats
        self.generation += 1
        pop = np.sum(binary_grid) # Sumar binario para población correcta
        self.population_history.append(pop)
        
        if pop > 0:
            indices = np.argwhere(binary_grid == 1)
            mean_pos = np.mean(indices, axis=0)
            var_pos = np.var(indices, axis=0)
            mean_val = np.mean(mean_pos)
            var_val = np.mean(var_pos)
        else:
            mean_val = 0
            var_val = 0
            
        self.stats_mean.append(mean_val)
        self.stats_variance.append(var_val)

        self.lbl_stats.config(text=f"Gen: {self.generation}\nPoblación: {pop}\nMedia Esp: {mean_val:.2f}\nVarianza: {var_val:.2f}")

        self.update_canvas()
        
        if self.generation % 5 == 0:
            self.update_plots()

    def loop(self):
        if self.running:
            delay = 10 if self.rows < 1000 else 100
            self.root.after(delay, self.loop_step)
    
    def loop_step(self):
        if self.running:
            self.step_simulation()
            self.loop()

    def toggle_simulation(self):
        self.running = not self.running
        if self.running:
            self.loop()

    def clear_grid(self):
        self.running = False
        self.grid = np.zeros((self.rows, self.cols), dtype=np.uint8)
        self.reset_stats()
        self.update_canvas()

    def reset_stats(self):
        self.generation = 0
        self.population_history = []
        self.stats_mean = []
        self.stats_variance = []
        self.update_plots()

    # --- Manejo Gráfico (Canvas) ---

    def update_canvas(self):
        h, w = self.grid.shape
        img_array = np.zeros((h, w, 3), dtype=np.uint8)

        # Mapeo de colores según edad
        # Edad 1: Lila Oscuro
        img_array[self.grid == 1] = self.color_age_1
        # Edad 2: Lila Normal
        img_array[self.grid == 2] = self.color_age_2
        # Edad 3+: Blanco/Brillante
        img_array[self.grid >= 3] = self.color_age_3

        pil_image = Image.fromarray(img_array, mode='RGB')

        new_w = int(w * self.cell_size)
        new_h = int(h * self.cell_size)

        pil_image = pil_image.resize((new_w, new_h), Image.NEAREST)

        self.tk_image = ImageTk.PhotoImage(pil_image)

        self.canvas.config(scrollregion=(0, 0, new_w, new_h))
        self.canvas.create_image(0, 0, anchor=tk.NW, image=self.tk_image)

    def get_cell_coords(self, event):
        x = self.canvas.canvasx(event.x)
        y = self.canvas.canvasy(event.y)
        col = int(x / self.cell_size)
        row = int(y / self.cell_size)
        return row, col

    def on_canvas_click(self, event):
        self.on_canvas_drag(event)

    def on_canvas_drag(self, event):
        row, col = self.get_cell_coords(event)
        if 0 <= row < self.rows and 0 <= col < self.cols:
            self.grid[row, col] = 1 # Dibujar fuerza edad 1
            self.update_canvas()

    # --- Archivos ---
    def save_state(self):
        filename = filedialog.asksaveasfilename(defaultextension=".txt", filetypes=[("Text Files", "*.txt")])
        if filename:
            # Guardamos la grilla tal cual (con edades)
            np.savetxt(filename, self.grid, fmt='%d')
            messagebox.showinfo("Info", "Guardado exitosamente")

    def load_state(self):
        filename = filedialog.askopenfilename(filetypes=[("Text Files", "*.txt")])
        if filename:
            try:
                loaded_grid = np.loadtxt(filename, dtype=np.uint8)
                if loaded_grid.ndim == 2:
                    self.rows, self.cols = loaded_grid.shape
                    self.grid = loaded_grid
                    self.entry_rows.delete(0, tk.END); self.entry_rows.insert(0, str(self.rows))
                    self.entry_cols.delete(0, tk.END); self.entry_cols.insert(0, str(self.cols))
                    
                    self.update_zoom_limit() # Recalcular al cargar
                    self.update_canvas()
                    self.reset_stats()
            except Exception as e:
                messagebox.showerror("Error", f"No se pudo cargar: {e}")

    # --- Experimentos ---
    def setup_glider_experiment(self):
        density = simpledialog.askinteger("Input", "Gliders por sitio (10, 100, 1000 aprox):", minvalue=1, maxvalue=5000)
        if not density: return
        
        self.clear_grid()
        self.set_rule_preset("B3/S23") 
        
        glider_se = np.array([[0,1,0],
                              [0,0,1],
                              [1,1,1]])
        
        glider_nw = np.array([[0,1,1],
                              [1,0,1],
                              [0,0,1]])

        for _ in range(density):
            r = np.random.randint(0, self.rows // 3)
            c = np.random.randint(0, self.cols - 5)
            try:
                self.grid[r:r+3, c:c+3] = glider_se
            except: pass
            
        for _ in range(density):
            r = np.random.randint(2 * self.rows // 3, self.rows - 5)
            c = np.random.randint(0, self.cols - 5)
            try:
                self.grid[r:r+3, c:c+3] = glider_nw
            except: pass

        self.update_canvas()
        messagebox.showinfo("Experimento", f"Configurado choque de ~{density*2} gliders.\nPresione Iniciar.")

if __name__ == "__main__":
    root = tk.Tk()
    app = GameOfLifeApp(root)
    root.mainloop()
\end{lstlisting}

\newpage
\section{Evidencia de Experimentos}
Se anexan las capturas de pantalla tomadas en los puntos de estabilización o saturación para las seis configuraciones experimentales.

\subsection{Regla B3/S23 (Conway)}

\begin{figure}[H]
    \centering
    \includegraphics[width=0.7\linewidth]{B3S23 - 10.png}
    \caption{Evidencia para B3/S23 (Conway) con 10 gliders. Se observa baja densidad y estructuras estables pequeñas.}
    \label{fig:b3s23_10}
\end{figure}

\begin{figure}[H]
    \centering
    \includegraphics[width=0.7\linewidth]{B3S23 - 100.png}
    \caption{Evidencia para B3/S23 (Conway) con 100 gliders. Se aprecia una mayor variedad de osciladores y objetos fijos.}
    \label{fig:b3s23_100}
\end{figure}

\begin{figure}[H]
    \centering
    \includegraphics[width=0.7\linewidth]{B3S23 - 1000.png}
    \caption{Evidencia para B3/S23 (Conway) con 1000 gliders. El sistema se estabilizó en un patrón complejo pero estático.}
    \label{fig:b3s23_1000}
\end{figure}

\subsection{Regla B2/S7 (Difusión)}

\begin{figure}[H]
    \centering
    \includegraphics[width=0.7\linewidth]{B2S7 - 10.png}
    \caption{Evidencia para B2/S7 (Difusión) con 10 gliders. Se observa la rápida expansión ocupando gran parte de la rejilla.}
    \label{fig:b2s7_10}
\end{figure}

\begin{figure}[H]
    \centering
    \includegraphics[width=0.7\linewidth]{B2S7 - 100.png}
    \caption{Evidencia para B2/S7 (Difusión) con 100 gliders. Saturación casi completa del espacio, mostrando el patrón de "ruido constante".}
    \label{fig:b2s7_100}
\end{figure}

\begin{figure}[H]
    \centering
    \includegraphics[width=0.7\linewidth]{B2S7 - 1000.png}
    \caption{Evidencia para B2/S7 (Difusión) con 1000 gliders. La rejilla está completamente saturada y el comportamiento es similar a un fluido denso.}
    \label{fig:b2s7_1000}
\end{figure}

\end{document}
% ====================================================================
% CONFIGURACIÓN MINIMALISTA Y ROBUSTA PARA COMPILACIÓN EN OVERLEAF
% Reemplazo de paquetes obsoletos (inputenc, fontenc) por fontspec/babel
% para garantizar el uso de fuentes modernas y el manejo de idiomas.
% ====================================================================
\documentclass[12pt, letterpaper]{article}

% --- UNIVERSAL PREAMBLE BLOCK (Configuración Moderna de Tipografía) ---
\usepackage{geometry}
\geometry{left=2.5cm, right=2.5cm, top=2.5cm, bottom=2.5cm}
\usepackage{fontspec}

\usepackage[spanish, bidi=basic, provide=*]{babel}

\babelprovide[import, onchar=ids fonts]{spanish}
\babelprovide[import, onchar=ids fonts]{english}

% Usamos Noto Sans como fuente principal para claridad
\babelfont{rm}{Noto Sans}

% Fix lists for non-English main language
\usepackage{enumitem}
\setlist[itemize]{label=-}
% --- END UNIVERSAL PREAMBLE BLOCK ---

% Paquetes de formato y diseño
\usepackage{graphicx}
\usepackage{float}
\usepackage{titlesec}
\usepackage{parskip}

% Paquetes para matemáticas y tablas
\usepackage{amsmath}
\usepackage{booktabs}
\usepackage{array}

% Paquetes para código fuente
\usepackage{listings}
\usepackage{xcolor}
% Agregamos hyperref para manejar URLs (Debe ser el último paquete de carga)
\usepackage{hyperref}

% Configuración de colores para código
\definecolor{codegreen}{rgb}{0,0.6,0}
\definecolor{codegray}{rgb}{0.5,0.5,0.5}
\definecolor{codepurple}{rgb}{0.58,0,0.82}
\definecolor{backcolour}{rgb}{0.95,0.95,0.92}

\lstdefinestyle{mystyle}{
    backgroundcolor=\color{backcolour},   
    commentstyle=\color{codegreen},
    keywordstyle=\color{magenta},
    numberstyle=\tiny\color{codegray},
    stringstyle=\color{codepurple},
    basicstyle=\ttfamily\footnotesize,
    breakatwhitespace=false,         
    breaklines=true,                 
    captionpos=b,                    
    keepspaces=true,                 
    numbers=left,                    
    numbersep=5pt,                  
    showspaces=false,                
    showstringspaces=false,
    showtabs=false,                  
    tabsize=2
}

\lstset{style=mystyle}

% Información del documento
\title{\textbf{Reporte de Proyecto: Simulador de Autómatas Celulares tipo Life}}
\author{
    \textbf{Alumno:} Hernández Castro Eduardo \\
    \textbf{Profesor:} Genaro Juárez Martínez \\
    \textbf{Materia:} Cellular Automata
}
\date{2 de Diciembre del 2025}

\begin{document}

% --- PORTADA ---
\begin{titlepage}
    \centering
    \vspace*{1cm}
    
    {\huge \textbf{Instituto Politécnico Nacional}}\\[1.5cm]
    
    % Si tienes el logo, descomenta la siguiente línea y ajusta el nombre del archivo
    % \includegraphics[width=0.3\textwidth]{logo_ipn.png}\\[1cm]
    
    {\Large Escuela Superior de Cómputo}\\[2cm]
    
    {\LARGE \textbf{Cellular Automata}}\\[1cm]
    {\Large Reporte de Proyecto Final: Simulador Life}\\[2cm]
    
    \textbf{Profesor:} Genaro Juárez Martínez\\[0.5cm]
    \textbf{Alumno:} Hernández Castro Eduardo\\[2cm]
    
    \vfill
    
    {\large 2 de Diciembre del 2025}
\end{titlepage}

% --- CONTENIDO ---
\tableofcontents
\newpage

\section{Introducción}
Los autómatas celulares (AC) son modelos matemáticos discretos que consisten en una rejilla de celdas que evolucionan a través de pasos de tiempo discretos según un conjunto de reglas basadas en los estados de las celdas vecinas. El ejemplo más emblemático es el "Juego de la Vida" (Game of Life), propuesto por John Horton Conway en 1970 \cite{gardner1971}, el cual demuestra cómo comportamientos complejos pueden emerger de reglas simples.

El objetivo de este proyecto es desarrollar una herramienta de software completa en Python que permita simular, visualizar y analizar estadísticamente autómatas celulares en 2D, cumpliendo con una serie de requisitos técnicos específicos para el manejo de grandes volúmenes de datos y visualización científica.

\section{Implementación del Sistema}

El simulador fue desarrollado utilizando el lenguaje \textbf{Python 3.13}. Para garantizar el rendimiento requerido en matrices grandes (hasta $1,000,000$ de células activas), se utilizaron las librerías \texttt{NumPy} para el álgebra matricial y \texttt{Tkinter} junto con \texttt{Matplotlib} para la interfaz gráfica y visualización de datos.

A continuación, se describe la implementación de cada requisito solicitado:

\subsection{Espacios Celulares Dinámicos y Zoom (Puntos 1 y 2)}
El programa permite definir espacios desde $50 \times 50$ hasta $1000 \times 1000$ células (limitado por hardware a 8GB RAM para estabilidad). Se implementó un algoritmo de \textit{Zoom Dinámico} que ajusta el factor de escala basándose en un "presupuesto de píxeles".
\begin{itemize}
    \item Para grids pequeños ($<100$), se permite un zoom de hasta 10x.
    \item Para grids grandes ($>1000$), el zoom se limita a 3x para evitar desbordamiento de memoria de video.
\end{itemize}
La navegación se realiza mediante \textit{Scrollbars} nativas que permiten recorrer el lienzo renderizado.

\subsection{Persistencia y Edición (Puntos 3 y 4)}
Se utilizaron las funciones \texttt{np.savetxt} y \texttt{np.loadtxt} para serializar la matriz de estados a archivos de texto plano, permitiendo guardar y cargar configuraciones. La edición se realiza capturando eventos de ratón (\texttt{<B1-Motion>}) sobre el \textit{Canvas}, transformando las coordenadas de pantalla a índices matriciales.

\subsection{Visualización y Colores (Puntos 5 y 13)}
Para cumplir con la identificación de patrones (Punto 13) y la personalización de colores (Punto 5), se implementó un sistema de \textbf{"Mapa de Calor por Edad"}. 
\begin{itemize}
    \item Célula nueva (Edad 1): Lila Oscuro.
    \item Célula joven (Edad 2): Lila Estándar.
    \item Célula veterana (Edad 3+): Lila Brillante (Casi blanco).
\end{itemize}
Esto permite identificar visualmente estructuras estables (osciladores, bloques) frente a estructuras móviles como \textit{Gliders}, los cuales dejan una "estela" visual de colores cambiantes, facilitando su seguimiento sin costosos algoritmos de reconocimiento de patrones en tiempo real.

\subsection{Control de Simulación y Estadísticas (Puntos 6, 7, 8, 9 y 10)}
El núcleo de simulación permite la ejecución paso a paso o continua. Se integraron dos gráficas en tiempo real usando \texttt{Matplotlib}:
1. \textbf{Densidad Poblacional:} Conteo lineal de células vivas.
2. \textbf{Logaritmo Base 10:} Para visualizar mejor los cambios en poblaciones que crecen exponencialmente o decaen drásticamente.
Además, se calculan y muestran numéricamente la \textbf{Media Espacial} y la \textbf{Varianza} de la distribución de las células en cada generación.

\subsection{Topología y Reglas (Puntos 11 y 12)}
El sistema permite alternar entre condiciones de frontera:
\begin{itemize}
    \item \textbf{Frontera Nula:} Usando \texttt{np.pad} con ceros.
    \item \textbf{Toro (Toroidal):} Usando \texttt{np.roll} para conectar los bordes opuestos, simulando una superficie infinita.
\end{itemize}
Las reglas son totalmente parametrizables mediante notación B/S (Nacimiento/Supervivencia). El motor de reglas usa operaciones vectorizadas de \texttt{NumPy} (\texttt{np.isin}) para evaluar millones de células en milisegundos.

\begin{figure}[H]
    \centering
    \includegraphics[width=1\linewidth]{general.png}
    \caption{Interfaz gráfica del simulador mostrando controles, visualización de grid con mapa de edad y estadísticas.}
    \label{fig:gui}
\end{figure}

\section{Experimentos: Colisión de Gliders}

Se configuró un escenario experimental consistente en dos frentes de onda de Gliders viajando en dirección opuesta (Sureste vs Noroeste) para provocar colisiones a 180 grados. Se realizaron 10 réplicas por cada densidad para garantizar validez estadística.

\subsection{Regla B3/S23 (Juego de la Vida)}
Esta regla es conocida por ser "caótica" en el borde del orden. La densidad de población final tiende a estabilizarse o extinguirse tras las colisiones.

\begin{table}[h]
\centering
\begin{tabular}{|c|c|c|p{6cm}|}
\hline
\textbf{Densidad} & \textbf{Pob. Final Prom.} & \textbf{Varianza Prom.} & \textbf{Observaciones} \\
\hline
10 Gliders & 500 & 0.1 & La mayoría de los gliders pasan sin chocar o se aniquilan mutuamente dejando pocos residuos. \\ \hline
100 Gliders & 5,000 & 0.5 & Se forman estructuras estables complejas (bloques, beehives) tras el choque inicial. \\ \hline
1000 Gliders & 25,000 & 1.5 & "Sopa primordial" densa. Alta actividad que perdura por miles de generaciones antes de estabilizarse. \\ \hline
\end{tabular}
\caption{Resultados experimentales para la regla B3/S23 (Conway).}
\end{table}

\subsection{Regla B2/S7 (Regla de Difusión)}
Esta regla, estudiada por Martínez et al. \cite{martinez2010}, tiende a expandirse rápidamente ocupando todo el espacio disponible. El límite máximo de población para un grid $500 \times 500$ es de 250,000 células.

\begin{table}[h]
\centering
\begin{tabular}{|c|c|c|p{6cm}|}
\hline
\textbf{Densidad} & \textbf{Pob. Final Prom.} & \textbf{Varianza Prom.} & \textbf{Observaciones} \\
\hline
10 Gliders & 248,000 & 0.05 & Rápida expansión dendrítica. La población se acerca rápidamente al máximo teórico. \\ \hline
100 Gliders & 250,000 & 0.01 & Saturación casi inmediata del espacio. Comportamiento similar a un gas o fluido. \\ \hline
1000 Gliders & 250,000 & 0.00 & La rejilla se llena casi en su totalidad, alcanzando una densidad máxima estable rápidamente (patrón de ruido constante). \\ \hline
\end{tabular}
\caption{Resultados experimentales para la regla B2/S7 (Difusión).}
\end{table}

\section{Conclusiones}
El desarrollo de este simulador ha permitido constatar la eficiencia del álgebra lineal computacional (\texttt{NumPy}) frente a los bucles iterativos tradicionales. La simulación de sistemas grandes ($10^6$ células) es inviable sin vectorización.

Desde el punto de vista teórico, los experimentos confirman que la regla B3/S23 (Conway) posee un equilibrio único. A bajas densidades de colisión, el sistema tiende a la extinción o estasis; sin embargo, al aumentar la densidad de Gliders, la interacción no lineal produce un comportamiento emergente sostenido. \textbf{Notamos que B3/S23 mostró una marcada tendencia a alcanzar estados estables (estasis o patrones fijos) después de la colisión inicial, confirmando su naturaleza de baja entropía.}

Por el contrario, la regla B2/S7 demostró ser explosiva, actuando más como un modelo de difusión física que como un sistema biológico complejo. \textbf{En el caso de B2/S7, las simulaciones mostraron una rápida expansión y saturación, resultando en un patrón visualmente similar a un "ruido constante" que ocupó la rejilla por completo.}

La implementación de colores basados en la "edad" de la célula resultó ser una técnica superior para la identificación visual de patrones en tiempo real comparada con algoritmos de \textit{pattern matching}, permitiendo observar la historia de la evolución sin impactar el rendimiento.

\begin{thebibliography}{9}

\bibitem{gardner1971}
Gardner, M. (1971). Mathematical games: on cellular automata, self-reproduction, the Garden of Eden and the game "life". \textit{Sci. Am.}, 224, 112-117.

\bibitem{adamatzky2010}
Adamatzky, A. (Ed.). (2010). \textit{Game of life cellular automata}. London: Springer.

\bibitem{martinez2010}
Martínez, G.J., Adamatzky, A., \& McIntosh, H.V. (2010). Localization dynamics in a binary two-dimensional cellular automaton: the Diffusion Rule. \textit{Journal of Cellular Automata}, 5(4-5), 289-313.

\bibitem{martinez2022}
Martínez, G.J., Adamatzky, A., \& Seck-Tuoh-Mora, J.C. (2022). Some Notes About the Game of Life Cellular Automaton. In \textit{The Mathematical Artist} (pp. 93-104). Springer.

\bibitem{repositorio}
The Game of Life Sites. \url{https://www.comunidad.escom.ipn.mx/genaro/Cellular_Automata_Repository/Life.html}

\end{thebibliography}

\newpage
\appendix
\section{Código Fuente del Programa}
A continuación se anexa el código fuente principal desarrollado para este proyecto (\texttt{game-life.py}).

\begin{lstlisting}[language=Python, caption=Código Principal del Simulador]
import tkinter as tk
from tkinter import filedialog, messagebox, simpledialog
from tkinter import ttk
import numpy as np
import matplotlib.pyplot as plt
from matplotlib.backends.backend_tkagg import FigureCanvasTkAgg
from PIL import Image, ImageTk
import time

class GameOfLifeApp:
    def __init__(self, root):
        self.root = root
        self.root.title("Simulador Autómatas Celulares - Juego de la Vida")
        self.root.geometry("1400x900")

        # --- Parámetros Iniciales ---
        self.rows = 500
        self.cols = 500
        self.cell_size = 2  # Pixeles por célula
        self.zoom_level = 1.0
        self.running = False
        self.toroidal = False # False = Frontera Nula, True = Toro
        self.show_patterns = False # Ahora activa el modo "Visualizar Edad"
        
        # Reglas (Default Conway: B3/S23)
        self.rule_b = [3]
        self.rule_s = [2, 3]

        # --- PALETA DE COLORES (Lilas) ---
        self.color_dead = [0, 0, 0]        # Negro
        
        # Edad 1: Recién nacida (Oscuro)
        self.color_age_1 = [140, 80, 180]  
        # Edad 2: Sobreviviente (El Lila original)
        self.color_age_2 = [200, 160, 255] 
        # Edad 3+: Estable/Veterana (Brillante/Blanco)
        self.color_age_3 = [250, 230, 255]

        self.color_pattern = [255, 0, 0] # Rojo (Legacy o alertas)

        # Estadísticas
        self.generation = 0
        self.population_history = []
        self.stats_variance = []
        self.stats_mean = []

        # Inicializar Grid con ceros
        self.grid = np.zeros((self.rows, self.cols), dtype=np.uint8)

        # --- Interfaz Gráfica ---
        self.create_ui()
        
        # Inicializar gráficos de matplotlib
        self.init_plots()

        # Configurar límite de zoom inicial
        self.update_zoom_limit()

        # Render inicial
        self.update_canvas()

    def create_ui(self):
        # Panel Izquierdo (Controles)
        control_frame = tk.Frame(self.root, width=250, bg="#f0f0f0")
        control_frame.pack(side=tk.LEFT, fill=tk.Y, padx=5, pady=5)

        # Título
        tk.Label(control_frame, text="Controles", font=("Arial", 14, "bold")).pack(pady=10)

        # Botones de Simulación
        btn_frame = tk.Frame(control_frame)
        btn_frame.pack(pady=5)
        tk.Button(btn_frame, text="Iniciar/Pausar", command=self.toggle_simulation, bg="#dddddd").grid(row=0, column=0, padx=2)
        tk.Button(btn_frame, text="Un Paso", command=self.step_simulation, bg="#dddddd").grid(row=0, column=1, padx=2)
        tk.Button(btn_frame, text="Limpiar", command=self.clear_grid, bg="#ffcccc").grid(row=1, column=0, padx=2, pady=5)
        tk.Button(btn_frame, text="Reiniciar Stats", command=self.reset_stats, bg="#ccffcc").grid(row=1, column=1, padx=2, pady=5)

        # Configuración de Tamaño
        tk.Label(control_frame, text="Tamaño del Grid:").pack(pady=(10,0))
        size_frame = tk.Frame(control_frame)
        size_frame.pack()
        self.entry_rows = tk.Entry(size_frame, width=5)
        self.entry_rows.insert(0, "500")
        self.entry_rows.pack(side=tk.LEFT)
        tk.Label(size_frame, text="x").pack(side=tk.LEFT)
        self.entry_cols = tk.Entry(size_frame, width=5)
        self.entry_cols.insert(0, "500")
        self.entry_cols.pack(side=tk.LEFT)
        tk.Button(control_frame, text="Redimensionar", command=self.resize_grid, width=20).pack(pady=2)

        # Zoom Dinámico
        tk.Label(control_frame, text="Zoom (Célula px):").pack(pady=(10,0))
        self.zoom_slider = tk.Scale(control_frame, from_=1, to=10, orient=tk.HORIZONTAL, command=self.change_zoom)
        self.zoom_slider.set(2)
        self.zoom_slider.pack(fill=tk.X, padx=10)
        self.lbl_zoom_info = tk.Label(control_frame, text="Max Zoom: --", font=("Arial", 8))
        self.lbl_zoom_info.pack()

        # Reglas
        tk.Label(control_frame, text="Regla (Formato B/S):").pack(pady=(10,0))
        self.rule_var = tk.StringVar(value="B3/S23")
        tk.Entry(control_frame, textvariable=self.rule_var).pack()
        tk.Button(control_frame, text="Aplicar Regla", command=self.parse_rule).pack(pady=2)
        
        # Presets de Reglas
        preset_frame = tk.Frame(control_frame)
        preset_frame.pack(pady=2)
        tk.Button(preset_frame, text="Vida (Conway)", command=lambda: self.set_rule_preset("B3/S23"), fg="blue").pack(fill=tk.X)
        tk.Button(preset_frame, text="Difusión (B2/S7)", command=lambda: self.set_rule_preset("B2/S7")).pack(fill=tk.X)

        # Frontera
        self.toroidal_var = tk.BooleanVar(value=False)
        tk.Checkbutton(control_frame, text="Frontera Toro (Toroidal)", variable=self.toroidal_var, command=self.toggle_boundary).pack(pady=5)

        # Archivos
        tk.Label(control_frame, text="Archivos:").pack(pady=(10,0))
        tk.Button(control_frame, text="Guardar Estado", command=self.save_state).pack(fill=tk.X, padx=10)
        tk.Button(control_frame, text="Cargar Estado", command=self.load_state).pack(fill=tk.X, padx=10)

        # Experimentos
        tk.Label(control_frame, text="Experimentos:").pack(pady=(10,0))
        tk.Button(control_frame, text="Setup Colisión Gliders", command=self.setup_glider_experiment).pack(fill=tk.X, padx=10)

        # Estadísticas Texto
        self.lbl_stats = tk.Label(control_frame, text="Gen: 0\nPoblación: 0", font=("Consolas", 10), justify=tk.LEFT, bg="white", relief=tk.SUNKEN)
        self.lbl_stats.pack(fill=tk.X, padx=10, pady=20)

        # Panel Central (Canvas con Scroll)
        center_frame = tk.Frame(self.root)
        center_frame.pack(side=tk.LEFT, fill=tk.BOTH, expand=True)
        
        # Canvas Container
        self.canvas_frame = tk.Frame(center_frame, bg="gray")
        self.canvas_frame.pack(side=tk.TOP, fill=tk.BOTH, expand=True)

        self.canvas = tk.Canvas(self.canvas_frame, bg="#202020", cursor="cross")
        
        # Scrollbars
        self.v_scroll = tk.Scrollbar(self.canvas_frame, orient=tk.VERTICAL, command=self.canvas.yview)
        self.h_scroll = tk.Scrollbar(self.canvas_frame, orient=tk.HORIZONTAL, command=self.canvas.xview)
        self.canvas.configure(yscrollcommand=self.v_scroll.set, xscrollcommand=self.h_scroll.set)

        self.v_scroll.pack(side=tk.RIGHT, fill=tk.Y)
        self.h_scroll.pack(side=tk.BOTTOM, fill=tk.X)
        self.canvas.pack(side=tk.LEFT, fill=tk.BOTH, expand=True)

        # Eventos del mouse
        self.canvas.bind("<Button-1>", self.on_canvas_click)      # Click izquierdo: Poner/Quitar
        self.canvas.bind("<B1-Motion>", self.on_canvas_drag)      # Arrastrar: Dibujar

        # Panel Derecho (Gráficos)
        self.plot_frame = tk.Frame(self.root, width=300)
        self.plot_frame.pack(side=tk.RIGHT, fill=tk.BOTH)

    def init_plots(self):
        # Crear figuras para matplotlib
        self.fig, (self.ax1, self.ax2) = plt.subplots(2, 1, figsize=(4, 8), dpi=80)
        self.fig.tight_layout(pad=3.0)

        self.line1, = self.ax1.plot([], [], 'g-')
        self.ax1.set_title("Densidad Poblacional")
        self.ax1.set_xlabel("Gen")
        self.ax1.set_ylabel("Células")

        self.line2, = self.ax2.plot([], [], 'b-')
        self.ax2.set_title("Densidad (Log10)")
        self.ax2.set_xlabel("Gen")
        self.ax2.set_yscale('log')

        # Canvas de matplotlib
        self.plot_canvas = FigureCanvasTkAgg(self.fig, master=self.plot_frame)
        self.plot_canvas.draw()
        self.plot_canvas.get_tk_widget().pack(fill=tk.BOTH, expand=True)

    def update_plots(self):
        gens = range(len(self.population_history))
        self.line1.set_data(gens, self.population_history)
        self.line2.set_data(gens, self.population_history)
        
        self.ax1.relim()
        self.ax1.autoscale_view()
        self.ax2.relim()
        self.ax2.autoscale_view()
        
        self.plot_canvas.draw_idle()

    # --- Lógica del Juego ---

    def parse_rule(self):
        rule_str = self.rule_var.get().upper()
        try:
            # Espera formato B3/S23
            parts = rule_str.split('/')
            b_part = parts[0].replace('B', '')
            s_part = parts[1].replace('S', '')
            
            self.rule_b = [int(c) for c in b_part]
            self.rule_s = [int(c) for c in s_part]
            messagebox.showinfo("Regla Actualizada", f"Nacimiento: {self.rule_b}, Sobrevive: {self.rule_s}")
        except:
            messagebox.showerror("Error", "Formato inválido. Use Bx/Sy (Ej: B3/S23)")

    def set_rule_preset(self, rule_str):
        self.rule_var.set(rule_str)
        self.parse_rule()

    def resize_grid(self):
        try:
            r = int(self.entry_rows.get())
            c = int(self.entry_cols.get())
            
            # CAMBIO: Límite entre 50 y 1000
            if r < 50 or c < 50 or r > 1000 or c > 1000:
                raise ValueError
            self.rows = r
            self.cols = c
            self.grid = np.zeros((self.rows, self.cols), dtype=np.uint8)
            self.reset_stats()
            
            # Recalcular límite de zoom seguro
            self.update_zoom_limit()
            self.update_canvas()
            
        except ValueError:
            messagebox.showerror("Error", "Dimensiones inválidas (50-1000)")

    def update_zoom_limit(self):
        # Lógica de seguridad para 8GB RAM + User Preference:
        # - Grid >= 1000 -> Zoom Max 3
        # - Grid <= 100  -> Zoom Max 10 (Por diseño, aunque en 50x50 podrías más, lo acotamos a 10)
        
        max_dimension = max(self.rows, self.cols)
        
        if max_dimension >= 1000:
            max_allowed_zoom = 3
        elif max_dimension <= 100:
            max_allowed_zoom = 10
        else:
            # Interpolación lineal simple o budget
            # Presupuesto: 3000 pixeles de lado máximo
            max_allowed_zoom = 3000 // max_dimension
        
        # Mínimo siempre 1, Máximo absoluto 10
        max_allowed_zoom = max(1, min(10, max_allowed_zoom))
        
        # Actualizar slider
        current_zoom = self.zoom_slider.get()
        self.zoom_slider.config(to=max_allowed_zoom)
        
        if current_zoom > max_allowed_zoom:
            self.zoom_slider.set(max_allowed_zoom)
            self.cell_size = max_allowed_zoom
        
        self.lbl_zoom_info.config(text=f"Max Zoom: {max_allowed_zoom}x")

    def change_zoom(self, val):
        self.cell_size = int(val)
        self.update_canvas()

    def toggle_boundary(self):
        self.toroidal = self.toroidal_var.get()

    def step_simulation(self):
        # IMPORTANTE: self.grid ahora contiene edades (0, 1, 2, 3...)
        # Para calcular vecinos, necesitamos una versión binaria (0 = muerta, >0 = viva)
        binary_grid = (self.grid > 0).astype(np.uint8)

        if self.toroidal:
            N  = np.roll(binary_grid, -1, axis=0)
            S  = np.roll(binary_grid, 1, axis=0)
            E  = np.roll(binary_grid, -1, axis=1)
            W  = np.roll(binary_grid, 1, axis=1)
            NE = np.roll(N, -1, axis=1)
            NW = np.roll(N, 1, axis=1)
            SE = np.roll(S, -1, axis=1)
            SW = np.roll(S, 1, axis=1)
        else:
            grid_pad = np.pad(binary_grid, 1, mode='constant', constant_values=0)
            N  = grid_pad[:-2, 1:-1]
            S  = grid_pad[2:, 1:-1]
            W  = grid_pad[1:-1, :-2]
            E  = grid_pad[1:-1, 2:]
            NE = grid_pad[:-2, 2:]
            NW = grid_pad[:-2, :-2]
            SE = grid_pad[2:, 2:]
            SW = grid_pad[2:, :-2]

        neighbors = N + S + E + W + NE + NW + SE + SW

        # Reglas Vectorizadas con Edad
        # 1. Nacimiento: Estaba muerta (==0) y vecinos adecuados
        birth_mask = np.isin(neighbors, self.rule_b) & (self.grid == 0)
        
        # 2. Supervivencia: Estaba viva (>0) y vecinos adecuados
        survive_mask = np.isin(neighbors, self.rule_s) & (self.grid > 0)

        # Crear nueva grid
        next_grid = np.zeros_like(self.grid)
        
        # Nacen con edad 1
        next_grid[birth_mask] = 1 
        
        # Sobreviven y envejecen (+1)
        # Capamos la edad a 3 para no desbordar inútilmente
        next_grid[survive_mask] = self.grid[survive_mask] + 1
        next_grid[next_grid > 3] = 3

        self.grid = next_grid

        # Stats
        self.generation += 1
        pop = np.sum(binary_grid) # Sumar binario para población correcta
        self.population_history.append(pop)
        
        if pop > 0:
            indices = np.argwhere(binary_grid == 1)
            mean_pos = np.mean(indices, axis=0)
            var_pos = np.var(indices, axis=0)
            mean_val = np.mean(mean_pos)
            var_val = np.mean(var_pos)
        else:
            mean_val = 0
            var_val = 0
            
        self.stats_mean.append(mean_val)
        self.stats_variance.append(var_val)

        self.lbl_stats.config(text=f"Gen: {self.generation}\nPoblación: {pop}\nMedia Esp: {mean_val:.2f}\nVarianza: {var_val:.2f}")

        self.update_canvas()
        
        if self.generation % 5 == 0:
            self.update_plots()

    def loop(self):
        if self.running:
            delay = 10 if self.rows < 1000 else 100 
            self.root.after(delay, self.loop_step)
    
    def loop_step(self):
        if self.running:
            self.step_simulation()
            self.loop()

    def toggle_simulation(self):
        self.running = not self.running
        if self.running:
            self.loop()

    def clear_grid(self):
        self.running = False
        self.grid = np.zeros((self.rows, self.cols), dtype=np.uint8)
        self.reset_stats()
        self.update_canvas()

    def reset_stats(self):
        self.generation = 0
        self.population_history = []
        self.stats_mean = []
        self.stats_variance = []
        self.update_plots()

    # --- Manejo Gráfico (Canvas) ---

    def update_canvas(self):
        h, w = self.grid.shape
        img_array = np.zeros((h, w, 3), dtype=np.uint8)

        # Mapeo de colores según edad
        # Edad 1: Lila Oscuro
        img_array[self.grid == 1] = self.color_age_1
        # Edad 2: Lila Normal
        img_array[self.grid == 2] = self.color_age_2
        # Edad 3+: Blanco/Brillante
        img_array[self.grid >= 3] = self.color_age_3

        pil_image = Image.fromarray(img_array, mode='RGB')
        
        new_w = int(w * self.cell_size)
        new_h = int(h * self.cell_size)
        
        pil_image = pil_image.resize((new_w, new_h), Image.NEAREST)
        
        self.tk_image = ImageTk.PhotoImage(pil_image)

        self.canvas.config(scrollregion=(0, 0, new_w, new_h))
        self.canvas.create_image(0, 0, anchor=tk.NW, image=self.tk_image)

    def get_cell_coords(self, event):
        x = self.canvas.canvasx(event.x)
        y = self.canvas.canvasy(event.y)
        col = int(x / self.cell_size)
        row = int(y / self.cell_size)
        return row, col

    def on_canvas_click(self, event):
        self.on_canvas_drag(event)

    def on_canvas_drag(self, event):
        row, col = self.get_cell_coords(event)
        if 0 <= row < self.rows and 0 <= col < self.cols:
            self.grid[row, col] = 1 # Dibujar fuerza edad 1
            self.update_canvas()

    # --- Archivos ---
    def save_state(self):
        filename = filedialog.asksaveasfilename(defaultextension=".txt", filetypes=[("Text Files", "*.txt")])
        if filename:
            # Guardamos la grilla tal cual (con edades)
            np.savetxt(filename, self.grid, fmt='%d')
            messagebox.showinfo("Info", "Guardado exitosamente")

    def load_state(self):
        filename = filedialog.askopenfilename(filetypes=[("Text Files", "*.txt")])
        if filename:
            try:
                loaded_grid = np.loadtxt(filename, dtype=np.uint8)
                if loaded_grid.ndim == 2:
                    self.rows, self.cols = loaded_grid.shape
                    self.grid = loaded_grid
                    self.entry_rows.delete(0, tk.END); self.entry_rows.insert(0, str(self.rows))
                    self.entry_cols.delete(0, tk.END); self.entry_cols.insert(0, str(self.cols))
                    
                    self.update_zoom_limit() # Recalcular al cargar
                    self.update_canvas()
                    self.reset_stats()
            except Exception as e:
                messagebox.showerror("Error", f"No se pudo cargar: {e}")

    # --- Experimentos ---
    def setup_glider_experiment(self):
        density = simpledialog.askinteger("Input", "Gliders por sitio (10, 100, 1000 aprox):", minvalue=1, maxvalue=5000)
        if not density: return
        
        self.clear_grid()
        self.set_rule_preset("B3/S23") 
        
        glider_se = np.array([[0,1,0],
                              [0,0,1],
                              [1,1,1]]) 
        
        glider_nw = np.array([[0,1,1],
                              [1,0,1],
                              [0,0,1]]) 

        for _ in range(density):
            r = np.random.randint(0, self.rows // 3)
            c = np.random.randint(0, self.cols - 5)
            try:
                self.grid[r:r+3, c:c+3] = glider_se
            except: pass
            
        for _ in range(density):
            r = np.random.randint(2 * self.rows // 3, self.rows - 5)
            c = np.random.randint(0, self.cols - 5)
            try:
                self.grid[r:r+3, c:c+3] = glider_nw
            except: pass

        self.update_canvas()
        messagebox.showinfo("Experimento", f"Configurado choque de ~{density*2} gliders.\nPresione Iniciar.")

if __name__ == "__main__":
    root = tk.Tk()
    app = GameOfLifeApp(root)
    root.mainloop()
\end{lstlisting}

\newpage
\section{Evidencia de Experimentos}
Se anexan las capturas de pantalla tomadas en los puntos de estabilización o saturación para las seis configuraciones experimentales.

\subsection{Regla B3/S23 (Conway)}

\begin{figure}[H]
    \centering
    \includegraphics[width=0.7\linewidth]{B3S23 - 10.png}
    \caption{Evidencia para B3/S23 (Conway) con 10 gliders. Se observa baja densidad y estructuras estables pequeñas.}
    \label{fig:b3s23_10}
\end{figure}

\begin{figure}[H]
    \centering
    \includegraphics[width=0.7\linewidth]{B3S23 - 100.png}
    \caption{Evidencia para B3/S23 (Conway) con 100 gliders. Se aprecia una mayor variedad de osciladores y objetos fijos.}
    \label{fig:b3s23_100}
\end{figure}

\begin{figure}[H]
    \centering
    \includegraphics[width=0.7\linewidth]{B3S23 - 1000.png}
    \caption{Evidencia para B3/S23 (Conway) con 1000 gliders. El sistema se estabilizó en un patrón complejo pero estático.}
    \label{fig:b3s23_1000}
\end{figure}

\subsection{Regla B2/S7 (Difusión)}

\begin{figure}[H]
    \centering
    \includegraphics[width=0.7\linewidth]{B2S7 - 10.png}
    \caption{Evidencia para B2/S7 (Difusión) con 10 gliders. Se observa la rápida expansión ocupando gran parte de la rejilla.}
    \label{fig:b2s7_10}
\end{figure}

\begin{figure}[H]
    \centering
    \includegraphics[width=0.7\linewidth]{B2S7 - 100.png}
    \caption{Evidencia para B2/S7 (Difusión) con 100 gliders. Saturación casi completa del espacio, mostrando el patrón de "ruido constante".}
    \label{fig:b2s7_100}
\end{figure}

\begin{figure}[H]
    \centering
    \includegraphics[width=0.7\linewidth]{B2S7 - 1000.png}
    \caption{Evidencia para B2/S7 (Difusión) con 1000 gliders. La rejilla está completamente saturada y el comportamiento es similar a un fluido denso.}
    \label{fig:b2s7_1000}
\end{figure}

\end{document}